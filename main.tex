\documentclass{article}
\usepackage[utf8]{inputenc}
\usepackage{amsmath}
\usepackage{graphicx}

\title{Assignment 8}
\author{Swapnil Sirsat}
\date{25 January 2021}

\begin{document}

\maketitle

\section{Question}
Obtain the inverse of following matrix using elementary operations\\
$\begin{matrix}
0 & 1 & 2\\
1 & 2 & 3\\
3 & 1 & 1\\
\end{matrix}$
\section{Solution}
We know that
\begin{gather*}
    AA^{-1} = I
\end{gather*}
therefore,
\begin{gather*}
\begin{matrix}
0 & 1 & 2\\
1 & 2 & 3\\
3 & 1 & 1\\
\end{matrix}A^{-1} = \begin{matrix}
1 & 0 & 0\\
0 & 1 & 0\\
0 & 0 & 1
\end{matrix}\\
R_1 \leftrightarrow R_2 , R_3 \rightarrow R_3 - 3R_2\\
\begin{matrix}
1 & 2 & 3\\ 
0 & 1 & 2\\
0 & -5 & -8\\
\end{matrix} A^{-1} = \begin{matrix}
0 & 1 & 0\\
1 & 0 & 0\\
0 & -3 & 1 \\
\end{matrix}\\
R_3 \rightarrow R_3 + 5R_2, R_1 \rightarrow R_1 - 2R_2\\
\begin{matrix}
1 & 0 & -1\\
0 & 1 & 2\\
0 & 0 & 2\\
\end{matrix} A^{-1} = \begin{matrix}
-2 & 1 & 0\\
1 & 0 & 0 \\
5 & -3 & 1\\
\end{matrix}\\
R_3 \rightarrow \frac{1}{2} R_3\\
\begin{matrix}
1 & 0 & -1\\
1 & 1 & 2\\
0 & 0 & 1\\
\end{matrix} A^{-1} = \begin{matrix}
-2 & 1 & 0 \\
1 & 0 & 0 \\
\frac{5}{2} & \frac{-3}{2} & \frac{1}{1}\\
\end{matrix}\\
\end{gather*}
\begin{gather*}
R_2 \rightarrow R_2 - 2R_3, R_1 \rightarrow R_1 + R_3\\
\begin{matrix}
1 & 0 & 0\\
0 & 1 & 0\\
0 & 0 & 1
\end{matrix} A^{-1} = \begin{matrix}
\frac{1}{2} & \frac{-1}{2} & \frac{1}{2}\\
-4 & 3 & -1\\
\frac{5}{2} & \frac{-3}{2} & \frac{1}{2}\\
\end{matrix}\\
\implies IA^{-1} = A^{-1} =  \begin{matrix}
\frac{1}{2} & \frac{-1}{2} & \frac{1}{2}\\
-4 & 3 & -1\\
\frac{5}{2} & \frac{-3}{2} & \frac{1}{2}\\
\end{matrix}
\end{gather*}
Thus,
\begin{gather*}
     A^{-1} =  \begin{matrix}
\frac{1}{2} & \frac{-1}{2} & \frac{1}{2}\\
-4 & 3 & -1\\
\frac{5}{2} & \frac{-3}{2} & \frac{1}{2}\\
\end{matrix}
\end{gather*}



\end{document}
